\documentclass[aspectratio=169,xcolor=dvipsnames]{beamer}
\usepackage{ods}
\usepackage{ods-figs}
\usepackage[cm]{sfmath}
%\usepackage{enumitem}

\setlength{\leftmargini}{0pt}


\title{Skiplist: Searching, Adding, and Removing}

\begin{document}

\begin{frame}
  \titlepage
\end{frame}

\begin{frame}
  \frametitle{Skiplist}
  \framesubtitle{Searching, adding, and removing}

  \begin{center}
     \includegraphics[width=\textwidth]{figs/skiplist-sample}
  \end{center}
  \begin{block}{Lemma}
     In a skiplist containing $n$ values, the expected length of
     the search path for any node $u$ in $L_0$ is at most $O(\log n)$.
  \end{block}

\end{frame}


\begin{frame}
  \frametitle{Skiplist}
  \framesubtitle{Searching, adding, and removing}

  \begin{center}
       \multiinclude[<+>][format=pdf,start=1,graphics={width=\textwidth}]{figs/skiplist-ii}%
  \end{center}

\end{frame}

\begin{frame}
  \frametitle{Skiplist}
  \framesubtitle{Searching, adding, and removing}

  \begin{block}{Theorem}
     A Skiplist implements the SSet interface.
    \begin{itemize}
       \item The $\mathsf{find}(x)$, $\mathsf{add}(x)$, and $\mathsf{remove}(x)$ operations each run in $O(\log n)$ expected time.
    \end{itemize}
  \end{block}

\end{frame}



\closing

\end{document}

